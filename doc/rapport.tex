\documentclass[10pt,a4paper,final]{report}
\usepackage[utf8]{inputenc}
\usepackage{amsmath}
\usepackage{amsfonts}
\usepackage{amssymb}
\author{Gustav Ernberg}
\title{Swing komponenter}
\begin{document}
\maketitle
\chapter{Captcha-komponent}
\section{Vad}
\pagebreak
\section{Hur}
\pagebreak
\section{Varför}
\pagebreak
\chapter{Kursmål}
Kursen går ut på att du ska lära dig principer och programmeringstekniker för avancerade grafiska användargränssnitt. Vi kallar detta för interaktionsprogrammering. 
Mål för Kunskap och förståelse:
\begin{itemize}
\item Redogöra för arkitekturen hos ett händelsebaserat fönstersystem med grafiska komponenter (widgets).;
\item Redogöra för olika designprinciper för grafiska användargränssnitt;
\item Redogöra för hur interaktionsprogrammering ingår i utvecklingsarbete med användbara system.;
\end{itemize}
Mål för Färdighet och förmåga:
\begin{itemize}
\item Använda ett aktuellt programmeringsspråk och klassbibliotek för interaktionsprogrammering, t ex Java Swing.;
\item Skapa egna komponenter (widgets) som en utökning av ett befintligt klassbibliotek.
\item Programmera dynamiska interaktionstekniker som drag n drop.;
\item Modellera och implementera ett enkelt fönstersystem.;
\item Tillämpa designmönster för interaktionstekniker.;
\item Arbeta med interaktionsprogrammering i kontexten av användbarhetsproblem.;
\end{itemize}
Mål för Värderingsförmåga och förhållningssätt:;
\begin{itemize}
\item Redogöra för kopplingen mellan klassbibliotek för interaktionsprogramming och principer för design av användargränssnitt.;
\end{itemize}


\end{document}